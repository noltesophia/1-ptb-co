\chapter{Motivation}
Large companies usually have structured information about their processes in the format of process models and other artifacts. A large amount of documents containing process-relevant information is however found in an unstructured format such as PDF containing a mixture of diagrams, tables, and free text. A series of qualitative interviews conducted with large enterprises in the DACH region suggests an order of magnitude of 1k to 100k process-relevant documents per company. That's why there is great interest in research and for businesses to develop an approach that can leverage this hidden value by automatically extracting relevant process information into \gls{bpm} software. Such an approach contains multiple components, among others converting textual process descriptions to standardized \acs{bpmn} 2.0 diagrams, which is the topic of this work. The rapid improvements of \acsp{llm} in the past year have given the topic renewed momentum. Moreover, there is generally more interest in the intersection topic of \acsp{llm} and \acs{bpm} \cite{large-process-models}.